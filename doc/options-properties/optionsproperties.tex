\documentclass[a4paper]{article}

\usepackage{color}
\usepackage{booktabs}
\definecolor{citeblue}{rgb}{0.4,0,.6}
\definecolor{refcolor}{rgb}{0,0.1,0.5}
\usepackage[pdftex,a4paper=true,colorlinks=true,bookmarks=false,
            linkcolor=refcolor,citecolor=citeblue,urlcolor=blue]{hyperref}
\usepackage{listings}

\def\lstinlinen{\lstinline[basicstyle=\normalsize\sffamily]}

\begin{document}

\section{Options}

\subsection{Option Types}
\label{subsec:option-types}

Recall that when a user declares an option in the DSL, they must
specify the type of the option. An option type determines the Java
type used to represent the option's value and, consequently, the
parsing mechanism used for the corresponding option.
\autoref{tbl:builtin-types} lists the built-in option
types. 

In order to extend the CLOPS DSL, new option types may be implemented and
registered for use. An \emph{option factory} defines the known set of options. Thus,
to extend the CLOPS system and provide additional types, a developer
extends the default option factory to make the system
aware of new developer-provided types. The implementation of new option types 
can extend the built-in types to reuse and refine their functionality, or one
may be written from the ground up, so long as the \lstinlinen|Option| interface
is satisfied.

\begin{table}[th]
  \centering
  \begin{tabular}{llll}
   \toprule 
    Name           & Inherits\quad\quad & Java type & Properties   \\
                   & from               &           &              \\
    \midrule
    \textit{basic} & None  & $T$      & \hyperref[prop:default]{default[T]}, \hyperref[prop:suffixregexp]{suffixregexp[string]}\\
                   &       &          & \hyperref[prop:sanitizeprefix]{sanitizeprefix[boolean]}\\
                   &       &          & \hyperref[prop:description]{description[string]}\\
                   &       &          & \hyperref[prop:aliases]{aliases[string]}\\
    \hline
    
    boolean        & basic & $boolean$& \hyperref[prop:allowarg]{allowarg[boolean]} \\
    \midrule
          
    counted-boolean& basic & $int$    & \hyperref[prop:countstart]{countstart[int]}, \hyperref[prop:countmax]{countmax[int]}, \\
                                  & & & \hyperref[prop:warnonexceedingmax]{warnonexceedingmax[boolean]}, \\
                                  & & & \hyperref[prop:erroronexceedingmax]{erroronexceedingmax[boolean]} \\
    \midrule
                                  
    string         & basic  & $String$ & \hyperref[prop:stripquotesifpresent]{stripquotesifpresent[boolean]}\\
    \midrule
    
    string-regexp  & string & $String$& \hyperref[prop:regexp]{regexp[string]} \\
    \midrule
    
    enum           & string & $enum$\footnote{Generated enum. See \ldots}
                                      & \hyperref[prop:choices]{choices[string]}, \hyperref[prop:casesensitive]{casesensitive[boolean]}
    \\
    \midrule
    
    int            & basic  & int     & \hyperref[prop:minvalue]{minvalue[int]}, \hyperref[prop:maxvalue]{maxvalue[int]} \\
    \midrule
    
    float          & basic  & $float$ & \hyperref[prop:minvalue]{minvalue[float]}, \hyperref[prop:maxvalue]{maxvalue[float]} \\
    \midrule
    
    file           & basic  & $File$  & \hyperref[prop:canexist]{canexist[boolean]}, \hyperref[prop:mustexist]{mustexist[boolean]}, \\
                                      & & & \hyperref[prop:canbedir]{canbedir[boolean]}, \hyperref[prop:mustbedir]{mustbedir[boolean]}, \\
                                      & & & \hyperref[prop:allowdash]{allowdash[boolean]} \\
    \midrule
                                      
    \textit{list}  & basic  & $List\langle T \rangle$ & \hyperref[prop:allowmultiple]{allowmultiple[boolean]}, \hyperref[prop:splitter]{splitter[string]} \\
    \midrule
     
    string-list    & list   & $List\langle String \rangle$ &  \\
    \midrule
    
    file-list      & list   & $List\langle File \rangle$ & \hyperref[prop:canexist]{canexist[boolean]}, \hyperref[prop:mustexist]{mustexist[boolean]},   \\
                   &        &                            & \hyperref[prop:canbedir]{canbedir[boolean]}, \hyperref[prop:mustbedir]{mustbedir[boolean]},   \\
                   &        &                            & \hyperref[prop:allowdash]{allowdash[boolean]}                       \\
    \midrule
    
    enum-list      & list   & $List\langle enum \rangle$ & \hyperref[prop:choices]{choices[string]}, \hyperref[prop:casesensitive]{casesensitive[boolean]} \\
    \midrule
    
    \bottomrule
  \end{tabular}
  \caption{Built-in option types and properties. Abstract types are typeset in \textit{italics}. $T$ is the type of the concrete option's value. }
  \label{tbl:builtin-types}
\end{table}

While parsing each option declaration, the DSL parser queries the option factory for 
the corresponding option type. The option factory provides the necessary details of
an option type: the class representing the option at runtime, as well as the Java type
that is used for the value of the option. 

The information provided by the option factory about each option's type is used in the code generation
phase. An interface is generated that provides access to the value for each option. For instance, for a
string option, a \emph{getter} method with return type $String$ is created. For each option there is also
a method with return type $boolean$ that determines whether an option has been set at all.

Each option type has a mechanism for indicating the properties that it accepts.
When parsing a given input specification, we can check that the option will
actually accept all provided properties. Assuming there are no problems, code is
generated that sets the option properties to their specified values during the
initialization phase.

\subsection{Properties}

%TODO alphabetize

\label{prop:default}
\paragraph{default} gives the option a default value of the correct type.

\label{prop:maxvalue}
\paragraph{maxvalue} gives an upper bound for the allowed value. Exceeding this will produce an error.

\label{prop:minvalue}
\paragraph{minvalue} gives a lower bound for the allowed value. Values lower than this will produce an error.

\label{prop:suffixregexp}
\paragraph{suffixregexp} TBD.

\label{prop:sanitizeprefix}
\paragraph{sanitizeprefix} TBD.

\label{prop:description}
\paragraph{description} TBD.

\label{prop:aliases}
\paragraph{aliases} TBD.

\label{prop:allowarg}
\paragraph{allowarg} TBD.

\label{prop:allowdash}
\paragraph{allowdash} TBD.

\label{prop:canbedir}
\paragraph{canbedir} TBD.

\label{prop:canexist}
\paragraph{canexist} TBD.

\label{prop:choices}
\paragraph{choices} TBD.

\label{prop:countstart}
\paragraph{countstart} TBD.

\label{prop:mustbedir}
\paragraph{mustbedir} TBD.

\label{prop:mustexist}
\paragraph{mustexist} TBD.

\label{prop:countmax}
\paragraph{countmax} TBD.

\label{prop:splitter}
\paragraph{splitter} TBD.

\label{prop:regexp}
\paragraph{regexp} TBD.

\label{prop:allowmultiple}
\paragraph{allowmultiple} TBD.

\label{prop:casesensitive}
\paragraph{casesensitive} TBD.

\label{prop:erroronexceedingmax}
\paragraph{erroronexceedingmax} TBD.

\label{prop:stripquotesifpresent}
\paragraph{stripquotesifpresent} TBD.

\label{prop:warnonexceedingmax}
\paragraph{warnonexceedingmax} TBD.


\end{document}
